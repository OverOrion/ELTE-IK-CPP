% !TeX root = 03gy_cpp
\documentclass[../cpp_book/cpp_book.tex]{subfiles}
\begin{document}
	\onlyinsubfile{
		\begin{center}
			{\LARGE\textbf{C++}}
			
			{\Large Gyakorlat jegyzet}
			
			3. óra
		\end{center}
		A jegyzetet \textsc{Umann} Kristóf készítette \textsc{Horváth} Gábor gyakorlata alapján. (\today)
	}

	\section{Statikus változók/függvények}
	A \texttt{static} kulcsszónak számos jelentése van, annak függvényében, hogy milyen kontextusban írjuk egy változó vagy függvény elé. 
	\subsection{Fordítási egységre lokális változók}
	A függvényeken és osztályokon kívül deklarált statikus változók az adott fordítási egységre lokálisak -- élettartamuk a futás elejétől végigtartamig tart, és kizárólagosan az adott fordítási egységben láthatóak.
	\medskip
	
	\fbox{\textbf{main.cpp}}
	\begin{lstlisting}
#include <iostream>

static int x;

int main()
{
	x = 2;
}
	\end{lstlisting}
	\medskip
	
	\fbox{\textbf{other.cpp}}
	\begin{lstlisting}
#include <iostream>

static int x;

void f()
{
	x = 0;
}
	\end{lstlisting}
	Ha ezt a két fájlt együtt fordítjuk, nem kapunk linkelési hibát, ugyanis a \texttt{main.cpp}-ben lévő \texttt{x} egy teljesen más változó, mint ami az \texttt{other.cpp}-ben van.
	
	\smallskip
	Csak úgy mint a globális változókra, fordítási egységen belül bármikor hivatkozhatunk egy statikusra, és hasonló módon inicializálódnak.
	\begin{lstlisting}
static int x;

int main()
{
	int x = 4;
	std::cout << ::x << std::endl; // 0
}
	\end{lstlisting}
	\subsection{Függvényen belüli statikus változók}
	Azokat a változókat, melyek függvényen belül vannak a \texttt{static} kulcsszóval definiálva, függvényszintű változónak is szokás hívni. Élettartamuk a függvény első hívásától a program futásának végéig tart, míg láthatóságuk csak az adott függvényen belül van. A hagyományos lokális változókkal ellenben tehát nem semmisülnek meg, amikor az adott függvény futása befejeződik. A következő kódrészlet szemlélteti ezt a viselkedést.
	
	\begin{lstlisting}
int f()
{
	static int x = 0;
	++x;
	return x;
}

int main() 
{
	for(int i = 0; i<5; i++)
		std::cout << f() << ' '; // 1 2 3 4 5
}
	\end{lstlisting}
	Ahogy az megfigyelhető fent, \texttt{x} csak egyszer inicializálódik, majd a későbbi függvényhívások után egyre növekszik az értéke.
	\subsection{Fordítási egységre lokális függvények}
  Nem csak változókat, függvényeket is deklarálhatunk statikusnak, melyek a fordítási egységre lokálisak. 
	\begin{lstlisting}
static int f()
{
	return 0;
}

int main() {std::cout << f();} // 0
	\end{lstlisting}
	Ezek a függvények csak az adott fordítási egységen belül érhetőek el.
	\begin{note}
		Figyelem! A később szóba kerülő metódusok esetében mást jelent a \texttt{static} kulcsszó, mint amit itt leírtunk.
	\end{note}
	\subsection{Névtelen/anonim névterek}
	Fordítási egységre lokális változókat és függvényeket tudunk deklarálni névtelen névterek (\textit{unnamed namespaces}, vagy \textit{anonymous namespaces}) segítségével. Egy név nélküli névteren belül deklarált változók és függvények hasonlóan viselkednek, mintha eléjük lenne írva a \texttt{static} kulcsszó.
	\begin{lstlisting}
namespace
{
	int x;
	std::string y;
	void f() {}
}
	\end{lstlisting}
	\begin{note}
    A \texttt{static} osztályon belüli jelentéséről később lesz szó.
	\end{note}
	\section{Függvény túlterhelés}
	Térjünk vissza a korábban megírt swap függvényünkhöz.
	\begin{lstlisting}
void swap(int &a, int &b)
{
	int tmp = a;
	a = b;
	b = tmp;
}
	\end{lstlisting}
	Ez a függvény addig jó, amíg csak \texttt{int}-eket szeretnénk megcserélni. Mi van, ha \texttt{std::string}-eket kéne? A megoldás egyszerű, \textbf{túlterheljük} (\textit{overload}) a \texttt{swap} függvényt.
	\begin{lstlisting}
void swap(std::string &a, std::string &b)
{
	std::string tmp = a;
	a = b;
	b = tmp;
}
	\end{lstlisting}
	Túlterhelésnek azt nevezzük, amikor két vagy több függvénynek a neve azonos, de a paramétereik különböznek. Tagfüggvényeket konstansság alapján is túl lehet terhelni.
	\begin{note}
		A később elhangzó osztályok tagfüggvényeinél a függvény konstanssága is számít (azonos nevű és paraméter listájú függvény különöző konstanssággal ugyanúgy túlterhelésnek számít).
	\end{note}
	\subsection{Operátor túlterhelés}
	Ha példaként vesszük a lineáris algebrából tanult rendezett valós számhármasokat ($\mathbb{R}^3$), lehetőségünk van arra, hogy a tanultak alapján definiáljuk a köztük értelmezett összeadást.
	%TODO temporális változók
	\begin{lstlisting}
struct LinAlgVector
{
	double x1, x2, x3;
};

LinAlgVector operator+(const LinAlgVector &lhs, const LinAlgVector &rhs)
{
	LinAlgVector ret;
	ret.x1 = lhs.x1 + rhs.x1;
	ret.x2 = lhs.x2 + rhs.x2; 
	ret.x3 = lhs.x3 + rhs.x3;
	return ret;
}

int main()
{
	LinAlgVector a, b;
	a.x1 = 1; a.x2 = 2; a.x3 = 3;
	b.x1 = 1; b.x2 = 1; b.x3 = 1;
	
	LinAlgVector c = a + b; 
}
	\end{lstlisting}
	A \texttt{main} függvényben lévő értékadás ezzel ekvivalens: \texttt{c = operator+(a, b)}, így láthatjuk, hogy az operátorok túlterhelése gyakorlatilag a függvénytúlterhelés speciális esete. 
	\smallskip 
	
	Írjuk meg a \texttt{print} függvényt 3D vektorokra! A gyakran kiíratáshoz használt jobb shift operátor (\textit{right shift operator}), a \texttt{\<} is túlterhelhető. Mivel mi az \texttt{std::cout} változóval szeretnénk majd kiíratni, melynek típusa \texttt{std::ostream}, így a függvényünk első paramétere egy ilyen típus lesz, a második meg egy \texttt{LinAlgVector} típus.
	
	\begin{lstlisting}
/* ... */

std::ostream& operator<<(std::ostream& os, const LinAlgVector &l)
{
	os << l.x1 << ' ' << l.x2 << ' ' << l.x3;
	return os;
}

int main()
{
	/* ... */
	std::cout << c << std::endl; // 2 3 4
}
	\end{lstlisting}
	
	Feltűnhet, hogy a stream objektumra mutató referenciát a függvény végén vissza is adjuk, hogy tudjuk a kiíratást láncolni.
\end{document}
