% !TeX root = 01exercise_cpp
\documentclass[../exercise_book/exercise_book.tex]{subfiles}
\begin{document}
	\onlyinsubfile{
		\begin{center}
			{\LARGE\textbf{C++}}
			
			{\Large Gyakorló feladatok}
			
			1. óra
		\end{center}
		A jegyzetet \textsc{Umann} Kristóf készítette. (\today)
	}
	\section{Előszó}
	Ez a jegyzet az ELTE Informatikai Kar hallgatóinak készült a \emph{Programozási Nyelvek C++} című tárgyhoz. Ez a dokumentum a gyakorlatokon történő számonkéréshez és vizsgához szolgál gyakorló feladatokkal.
	
	\smallskip
	Ez a dokumentum segédanyagként szolgál egy meglevő jegyzethez, mely szintén e projektnek a része!
	
	\smallskip
	A jegyzet első számú forrása \textsc{Horváth} Gábor 2015/2016/2 és 2016/2017/1 félévében tartott gyakorlatai, \textsc{Pataki} Norbert 2015/2016/1 előadásai, valamint a gyakorlaton megjelent helyettesítő tanárok órái: \textsc{Porkoláb} Zoltán, \textsc{Brunner} Tibor.
	
	Külön köszönet jár \textsc{Horváth} Gábornak, aki a jegyzet javításában segített és aktívan segít, és mindenki másnak, aki az esetleges hibák észrevétele után szóltak.
	
	\smallskip
	A jegyzet teljes mértékben nyílt forráskódú, amennyiben esetleges hibába, pontatlanságba botlana, vagy szeretne segíteni az jegyzet fejlesztésében, az alábbi linken megteheti:
	
	\url{https://github.com/Szelethus/ELTE-IK-CPP}.
	
	(Felhívnánk rá a figyelmet, hogy a szerkesztés jelen pillanatában a jegyzet nem teljes terjedelmében lektorált.)
	
	\section{Preprocesszor}
	\begin{exercise}
		Legyen adott ez a C++ fájl:
		
		\smallskip
		\fbox{\textbf{01\_01pp\_main.cpp}}
		\begin{lstlisting}
#include "01_01pp.h"
#include <iostream>

int main() { std::cout << add(5, 6) << std::endl; }
		\end{lstlisting}
		Kimenet: \texttt{11}
		
		\smallskip
		Implementáljuk a \texttt{01\_01pp.h} header fájlt mely tartalmazza az \texttt{add} függvény deklarációját, mely két egész számot vár paraméterül, és az összegüket adja vissza visszatérési értékeként! A függvényt definiáljuk a \texttt{01\_01pp.cpp} fájlban, majd fordítsuk le a programot és ellenőrizzük hogy helyesen működik-e!
	\end{exercise}
	\begin{exercise}
		Legyen adott ez a C++ fájl:
		
		\smallskip
		\fbox{\textbf{01\_02pp\_main.cpp}}
		\begin{lstlisting}
#include <iostream>

// *

int main() { std::cout << 5 * ADD(1, 2) << std::endl; }
		\end{lstlisting}
		Kimenet: \texttt{15}
		
		\smallskip
		Implementáljunk a \texttt{*}-al jelölt sorba egy preprocesszor függvényt, mely az összeadást valósítja meg! Figyeljünk a műveletek asszociativitására is!
	\end{exercise}
\end{document}
