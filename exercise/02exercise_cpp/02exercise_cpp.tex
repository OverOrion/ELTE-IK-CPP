% !TeX root = 02exercise_cpp
\documentclass[../exercise_book/exercise_book.tex]{subfiles}
\begin{document}
	\onlyinsubfile{
		\begin{center}
			{\LARGE\textbf{C++}}
			
			{\Large Gyakorló feladatok}
			
			2. óra
		\end{center}
		A jegyzetet \textsc{Umann} Kristóf készítette. (\today)
	}
	\section{Mutatók}
	
	\begin{exercise}
		Legyen adott ez a C++ fájl:
		
		\smallskip
		\fbox{\textbf{02\_01ptr\_main.cpp}}
		\begin{lstlisting}
#include <iostream>

int main() {
  int t[] = {1, 2, 3, 4};

  int *ptr;
  for (/* ... */) {
    std::cout << /* t egyik eleme */ << std::endl;
  }
}
		\end{lstlisting}
		Iteráljunk végig \texttt{t} tömb elemein újabb változó létrehozása nélkül! Hogyan változna a megoldás, ha \texttt{t} méretét megváltoztatjuk?
	\end{exercise}
	\begin{exercise}
		Legyen adott ez a C++ fájl:
		
		\smallskip
		\fbox{\textbf{02\_02ptr\_main.cpp}}
		\begin{lstlisting}
#include <iostream>

void changePointer(/* ... */) {
  // ...
}

int main() {
  int c = 5, d = 6;
  int *ptr = &c;
  changePointer(&ptr, &d);
  std::cout << ptr << '\n' << &d << std::endl;
}
		\end{lstlisting}
		Implementáljuk a \texttt{changePointer} függvényt, mely egy \texttt{int}re mutatót egy, a paraméterként átadott \texttt{int} objektum címére állít! Megoldásunk helyességét azzal ellenőrizhetjük, hogy \texttt{ptr} és \texttt{\&d} értékét kiíratjuk -- amennyiben ezek azonosak, az azt jelenti hogy \texttt{ptr} \texttt{d}-re mutat.
	\end{exercise}
	\begin{exercise}
		Miben változna az előző feladat megoldása, ha \texttt{c} és \texttt{d} konstansok lennének? Implementáljuk a \texttt{02\_03ptr\_main.cpp} fájlba!
	\end{exercise}
	\begin{exercise}
		Legyen adott ez a C++ fájl:
		
		\smallskip
		\fbox{\textbf{02\_04ptr\_main.cpp}}
		\begin{lstlisting}
#include <iostream>

int findIf(/* ... */) {
  // ...
}

bool isEven(int i) { return i % 2 == 0; }

bool isEqualToSeven(int i) { return i == 7; }

bool isNegative(int i) { return i < 0; }

int main() {
  int t[] = {1, 3, 4, 6, 7, 8, 10};

  std::cout << findIf(/* ... */, isEven) << ' '
            << findIf(/* ... */, isEqualToSeven) << ' '
            << findIf(/* ... */, isNegative) << std::endl;
}
		\end{lstlisting}
		Kimenet: \texttt{2 4 -1}
		
		\smallskip
		Valósítsuk meg a \texttt{findIf} függvényt, mely egy tetszőleges \texttt{int}-eket tartalmazó tömböt és egy predikátum függvényt (olyan függvény, melynek \texttt{bool} a visszatérési értéke és egy paramétere van, ez esetben \texttt{int} típusú) kap paraméterül, és amennyiben a tömb tartalmaz olyan elemet amelyre a predikátumfüggvény igazat ad, adja vissza a tömbbéli indexét, különben térjen vissza \texttt{-1}-el!
		
		Azt, hogy az Olvasó hogyan szeretné a \texttt{findIf} függvény implementálni (beleértve azt is, hogy milyen paramétereket várjon) nincs megkötve, azzal a kivétellel, hogy a predikátumfüggvény függvény legyen az utolsó paraméter.
	\end{exercise}
	\begin{exercise}
		Hogyan tudnánk a fenti feladatokat referenciákkal megoldani?
	\end{exercise}
\end{document}
