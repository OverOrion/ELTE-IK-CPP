\documentclass[a4paper,11.5pt,table]{article}
\usepackage[textwidth=170mm, textheight=230mm, inner=20mm, top=20mm, bottom=30mm]{geometry}
\usepackage[normalem]{ulem}
\usepackage[utf8]{inputenc}
\usepackage[T1]{fontenc}
\PassOptionsToPackage{defaults=hu-min}{magyar.ldf}
\usepackage[magyar]{babel}
\usepackage{amsmath, amsthm, amssymb, paralist, tikz, multirow}
\usetikzlibrary{arrows, positioning}

\usepackage{listings}
\lstdefinestyle{customc}{
	belowcaptionskip=1\baselineskip,
	breaklines=true,
	frame=L,
	language=C++,
	showstringspaces=false,
	basicstyle=\ttfamily,
	%identifierstyle=\color{blue},
	keywordstyle=\bfseries\color{green!40!black},
	stringstyle=\color{orange},
	emph = {std, A, B, BDIT, Base, BaseOne, BaseTwo, C, CharT, Circle, Compr, ConstIterator, Cont, D, Deltoid, Derived, Derived2, DerivedLast, DerivedOne, DerivedTwo, Fact, Fiu, Foo, Greater, Hallgato, IT, InputIt, Iterator, Kor, Lany, Less, LessByX, LinAlgVector, List, Matrix, Negyszog, Negyzet, RAIT, Rombusz, S, Sielo, Sikidom, StringLength, T, T1, T2, Templ, U, Val, X, Y, Z, bad_cast, bidirectional_iterator_tag, const_iterator, deque, forward_iterator_tag, input_iterator_tag, iostream, istream, iterator, iterator_category, line_editor, list, map, multimap, multiset, ostream, pair, random_access_iterator_tag, set, string, strlen, strlenWrong, value_type, vector}, 
	emphstyle = \color{blue},
	tabsize = 4
}

\lstdefinestyle{customasm}{
	belowcaptionskip=1\baselineskip,
	frame=L,
	language=[x86masm]Assembler,
	basicstyle=\ttfamily,
	commentstyle=\itshape\color{purple!40!black},
}

\lstset{
	escapechar=@,
	style=customc,
	literate =  {á}{{\'a}}1 {é}{{\'e}}1 {í}{{\'i}}1 {ó}{{\'o}}1 {ú}{{\'u}}1
	{Á}{{\'A}}1 {É}{{\'E}}1 {Í}{{\'I}}1 {Ó}{{\'O}}1 {Ú}{{\'U}}1
	{ö}{{\"o}}1 {ü}{{\"u}}1 {Ö}{{\"O}}1 {Ü}{{\"U}}1
	{ű}{{\H{u}}}1 {Ű}{{\H{U}}}1 {ő}{{\H{o}}}1 {Ő}{{\H{O}}}1
	{€}{{\euro}}1 {£}{{\pounds}}1	
}

\usepackage{hyperref}

\begin{document}
	%%%%%%%%%%%RÖVIDÍTÉSEK%%%%%%%%%%
	\setlength\parindent{0pt}
	\def\<{<\hspace{0mm}<}
	
	\theoremstyle{definition}
	\newtheorem{note}{Megjegyzés}[subsection]
  \newtheorem{example}{Példa}[subsection]
  \newtheorem{definition}{Definíció}[subsection]
	%%%%%%%%%%%%%%%%%%%%%%%%%%%%%%%%%%%%%%%%%%%%%%%%%%%%%%%%%%%%%%%%%%%%%

	
	\begin{center}
		{\LARGE\textbf{C++}}
		
		{\Large Gyakorlat jegyzet}
		
		3 óra.
	\end{center}
	A jegyzetet \textsc{Umann} Kristóf készítette \textsc{Porkoláb} Zoltán és \textsc{Horváth} Gábor gyakorlata alapján. (\today)
%	\section{Kifejezések kiértékelése}
%	Ugorjunk vissza egy korábbi példára.
%	\begin{lstlisting}
%#include <iostream>
%
%int main()
%{
%	int i = 0;
%	std::cout << i << i++ << std::endl;
%}
%	\end{lstlisting}
%	%\< egy normálisabban kinéző << operátort eredményez (definíció fent)
%	
%	Világos, hogy a fenti kódban nem definiált viselkedés szerepel. Már volt szó arról is, hogy a kiíratáshoz használt \texttt{\<} szimbólum az egy operátor, és az \texttt{<iostream>} könyvtárban található hozzá egy olyan overload, melynek egyik paramétere \texttt{std::ostream\&}, a másik pedig \texttt{int}. Ahogy korábban is említettük, az első paramétert adja vissza, így tudjuk a kiíratást láncolni. A fent leírt kiíratás ezzel lesz ekvivalens:
%	\begin{lstlisting}
%std::cout.operator<<(i).operator<<(i++);
%	\end{lstlisting}
%	
%	Itt látható egy \texttt{\<} operátor, ami így is felírható: \texttt{std::cout.operator$<<$(i)}. Ennek a függvényhívásnak van visszatérési értéke, méghozzá \texttt{std::cout}, ez teszi lehetővé a láncolást. Ez itt egy tagfüggvény, mely majdnem minden alaptípusra túl van terhelve. 
%	
%	Az biztos, hogy a második kiírt szám értéke 2. De az, hogy az első mennyi, nem definiált.
%	
%	\begin{example}
%		\texttt{x = k + 2}\quad \quad \texttt{y = k + 2}
%	\end{example}
%	
%	Ebben a példában (jó eséllyel) a fordító optimalizál, és \texttt{k+2}-t csak egyszer számolja ki. A C++ban szabvány által nem definiált viselkedéseknek köszönhetően sokkal hatékonyabb programokat kaphatunk, mert a fordítónak nagy szabadsága van abban, hogyan optimalizáljon.
%	
%	\begin{example}
%		Töltsük fel egy tömb elemeit növekvő sorrendben számokkal.
%		
%		\begin{lstlisting}
%int i = 0;
%int t[10];
%while (i < 10)
%{
%	t[i] = i++;
%}
%		\end{lstlisting}
%		
%		A kód viselkedése ismét nem definiált. Hiába van ott egy \textbf{post-fix} \texttt{++}\quad operator. Az, hogy az értékadás melyik oldalán levő \texttt{i} értékelődik ki először, nem specifikált.
%	\end{example}
%	
%	A szekvenciapontok korrekt használatával lehet kijavítani a fenti kódot.
%	
%	\begin{definition}
%		(Szekvenciapont) meghatározza a kifejezések kiértékelési sorrendjét futási időben. A szekvenciapont előtti kifejezéseknek hamarabb ki kell értékelődnie, mint a szekvenciapont utániaknak. Néhány példa szekvenciapontokra: vessző, \texttt{\&\&}, \texttt{||}, \texttt{?\quad :} 
%	\end{definition}
%	\begin{example}
%		\begin{lstlisting}
%f(i), ++i;
%i++<10 && f(i);
%i++<10 || f(i);
%i++<10 ? f(i) : g(i);
%		\end{lstlisting}
%		A fenti példák eredményei mind jól definiáltak, a szekvenciapontok miatt egyértelmű a kifejezések eredménye.
%		\begin{lstlisting}
%f(i++, j++);
%		\end{lstlisting}
%		Itt azonban az, hogy i vagy j értéke növekszik-e meg először, az nem specifikált. Bár valóban található ott vessző, de a vessző operátor mint szekvenciapont nem ekvivalens a függvény paramétereit elválasztó vesszővel.
%	\end{example}
%	
%	\begin{lstlisting}
%int f() {cout << 'f'; return 2;}
%int g() {cout << 'g'; return 1;}
%int h() {cout << 'h'; return 0;}
%	\end{lstlisting}
%	Mi fog történni \texttt{f() == g() == h()} kód írásakor?
%	
%	Itt a kimenet attól is függ, hogy milyen sorrendben értékelődnek ki az egyenlőség-vizsgáló operátorok. Az operátoroknak van megadott precedenciájuk, ami meghatározza mennyire erősen kötnek: erős például a pont, nyíl, [], gyengébb ennél a dereferencia, és így tovább. Azonban az azonos precedenciájú operátoroknál kérdéses, hogy a részkifejezések milyen sorrendben értékelődnek ki. 
%	
%	Visszatérve a fenti példára, a végrehajtási sorrend:
%	\texttt{(f() == g()) == h()}. Azaz, a \texttt{==} operátor balról jobbra zárójelezendő, tehát balasszociatív. Milyen sorrendben lesznek kiírva a karakterek? Ez nem specifikált, hiszen a függvényhívásokat nem választja el egymástól szekvenciapont.
%	
%	Van ahol más a zárójelezés, pl. \texttt{!++*++p}. Itt Először előrelépünk a p pointerrel, dereferáljuk, megnöveljük az értékét, és negáljuk. \texttt{!(++(*(++p)))}. Ilyen példa szintén az egyenlőség operátor: \texttt{x = y = z = 3.14}. 
%	\begin{note}
%		Bővebben: \url{http://en.cppreference.com/w/cpp/language/operator_precedence}
%	\end{note}
%	\begin{lstlisting}
%#include <iostream>
%
%const char* answer (const char *q);
%
%int main()
%{
%	std::cout << answer("Hogy vagy?") << answer("Biztos?") << std::endl;
%	return 0;
%}
%
%const char* answer (const char *q)
%{
%	std::cout << q;
%	static char buffer[80];
%	std::cin.getline(buffer,80);
%	return buffer;
%}
%	\end{lstlisting}
%	A fenti kódban előfordulhat, hogy a két üzenet kiírásának a sorrendje nem megfelelő. 
%	
%	A pontosvessző szekvenciapont, így a megoldás egyszerű:
%	
%	\texttt{std::cout $<<$ answer("Hogy vagy?");}
%	
%	\texttt{std::cout $<<$ answer("Biztos?");}
%	
%	A fenti kódban nem szép, hogy van egy statikus változó a függvényen belül. Azért statikus, hogy a buffer ne szűnjön meg a függvényből való visszatérés után, hiszen abban van a válasz. Egy statikus változó viszont életben van a program befejezéséig, ráadásul az esetleges párhuzamosítást is megakadályozza a későbbiekben.
%	\medskip
%	
%	A dinamikusan lefoglalt memória kezelése a stacken lévő objektumokkal szemben a programozó felelőssége. Nekünk kell lefoglalni, és ha nincs már rá szükségünk, nekünk is kell felszabadítani. C++11ben smart-pointerekkel ezt valamelyest automatizálhatjuk.
%	
%	\begin{lstlisting}
%#include <iostream>
%
%const char* answer (const char *q);
%
%int main()
%{
%	std::cout << answer("Hogy vagy?");
%	std::cout << answer("Biztos?");
%	return 0;
%}
%
%const char* answer (const char *q)
%{
%	std::cout << q;
%	char* buffer = new char[80];
%	std::cin.getline(buffer,80);
%	return buffer;
%}
%	\end{lstlisting}
%	A fenti kódban sajnos a lefoglalt memória elszivárgott. A \texttt{new} kulcsszóval létrehoztunk a dinamikus tárhelyen egy új változót, de azt soha nem szabadítottuk fel. Egy lehetőség lenne az, hogy referencia szerint átadunk egy buffert, amiben el tudjuk tárolni a választ. Annak azonban kényelmetlen a használata. Ennél szebb megoldás, ha az \texttt{std::string}-et használjuk.
%	\begin{lstlisting}
%#include <iostream>
%#include <string>
%
%std::string answer (std::string q);
%
%int main()
%{
%	std::cout << answer("Hogy vagy?");
%	std::cout << answer("Biztos?");
%	return 0;
%}
%
%std::string answer (std::string q)
%{
%	std::cout << q;
%	std::string buffer;
%	std::cin >> buffer;
%	return buffer;
%}
%	\end{lstlisting}
%	
%	Ez a memóriában úgy néz ki, hogy a stacken létrejön egy pointer ami egy heapen foglalt bufferre mutat. Az objektum másolásakor a heapen lévő buffer is másolásra kerül, az objektum megsemmisülésekor a heapen lévő buffer felszabadul. A másolások költségesek lehetnek, azonban a fordító optmializációi és a C++11-es move szemantika sokat javít a hatékonyságon.
%	

	\section{Statikus változók/függvények}
	A \texttt{static} kulcsszónak számos jelentése van, annak függvényében, hogy milyen kontextusban írjuk egy változó vagy függvény elé. 
	\subsection{Fordítási egységre lokális változók}
	A függvényeken és osztályokon kívül deklarált statikus változók az adott fordítási egységre lokálisak -- élettartamuk a futás elejétől végigtartamig tart, és kizárólagosan az adott fordítási egységben láthatóak.
	\medskip
	
	\fbox{\textbf{main.cpp}}
	\begin{lstlisting}
#include <iostream>

static int x;

int main()
{
	x = 2;
}
	\end{lstlisting}
	\medskip
	
	\fbox{\textbf{other.cpp}}
	\begin{lstlisting}
#include <iostream>

static int x;

void f()
{
	x = 0;
}
	\end{lstlisting}
	Ha ezt a két fájlt együtt fordítjuk, nem kapunk linkelési hibát, ugyanis a \texttt{main.cpp}-ben lévő \texttt{x} egy teljesen más változó, mint ami az \texttt{other.cpp}-ben van.
	
	\smallskip
	Csak úgy mint a globális változókra, fordítási egységen belül bármikor hivatkozhatunk egy statikusra, és hasonló módon inicializálódnak.
	\begin{lstlisting}
static int x;

int main()
{
	int x = 4;
	std::cout << ::x << std::endl; // 0
}
	\end{lstlisting}
	\subsection{Függvényen belüli statikus változók}
	Azokat a változókat, melyek függvényen belül vannak a \texttt{static} kulcsszóval definiálva, függvényszintű változónak is szokás hívni. Élettartamuk a függvény első hívásától a program futásának végéig tart, míg láthatóságuk csak az adott függvényen belül van. A hagyományos lokális változókkal ellenben tehát nem semmisülnek meg, amikor az adott függvény futása befejeződik. A következő kódrészlet szemlélteti ezt a viselkedést.
	
	\begin{lstlisting}
int f()
{
	static int x = 0;
	++x;
	return x;
}

int main() 
{
	for(int i = 0; i<5; i++)
		std::cout << f() << ' '; // 1 2 3 4 5
}
	\end{lstlisting}
	Ahogy az megfigyelhető fent, \texttt{x} csak egyszer inicializálódik, majd a későbbi függvényhívások után egyre növekszik az értéke.
	\subsection{Fordítási egységre lokális függvények}
  Nem csak változókat, függvényeket is deklarálhatunk statikusnak, melyek a fordítási egységre lokálisak. 
	\begin{lstlisting}
static int f()
{
	return 0;
}

int main() {std::cout << f();} // 0
	\end{lstlisting}
	Ezek a függvények csak az adott fordítási egységen belül érhetőek el.
	\begin{note}
		Figyelem! A később szóba kerülő metódusok esetében mást jelent a \texttt{static} kulcsszó, mint amit itt leírtunk.
	\end{note}
	\subsection{Névtelen/anonim névterek}
	Fordítási egységre lokális változókat és függvényeket tudunk deklarálni névtelen névterek (\textit{unnamed namespaces}, vagy \textit{anonymous namespaces}) segítségével. Egy név nélküli névteren belül deklarált változók és függvények hasonlóan viselkednek, mintha eléjük lenne írva a \texttt{static} kulcsszó.
	\begin{lstlisting}
namespace
{
	int x;
	std::string y;
	void f() {}
}
	\end{lstlisting}
	\begin{note}
    A \texttt{static} osztályon belüli jelentéséről később lesz szó.
	\end{note}
	\section{Függvény túlterhelés}
	Térjünk vissza a korábban megírt swap függvényünkhöz.
	\begin{lstlisting}
void swap(int &a, int &b)
{
	int tmp = a;
	a = b;
	b = tmp;
}
	\end{lstlisting}
	Ez a függvény addig jó, amíg csak \texttt{int}-eket szeretnénk megcserélni. Mi van, ha \texttt{std::string}-eket kéne? A megoldás egyszerű, \textbf{túlterheljük} (\textit{overload}) a \texttt{swap} függvényt.
	\begin{lstlisting}
void swap(std::string &a, std::string &b)
{
	std::string tmp = a;
	a = b;
	b = tmp;
}
	\end{lstlisting}
	Túlterhelésnek azt nevezzük, amikor két vagy több függvénynek a neve azonos, de a paramétereik különböznek. Tagfüggvényeket konstansság alapján is túl lehet terhelni.
	\begin{note}
		A később elhangzó osztályok tagfüggvényeinél a függvény konstanssága is számít (azonos nevű és paraméter listájú függvény különöző konstanssággal ugyanúgy túlterhelésnek számít).
	\end{note}
	\subsection{Operátor túlterhelés}
	Ahogy láttuk a 2. gyakorlat elején, lehetőségünk van függvényeket túlterhelni. Ez operátorokra is igaz. Ha példaként vesszük a lineáris algebrából tanult rendezett valós számhármasokat ($\mathbb{R}^3$), lehetőségünk van arra, hogy a tanultak alapján definiáljuk a köztük értelmezett összeadást.
	%TODO temporális változók
	\begin{lstlisting}
struct LinAlgVector
{
	double x1, x2, x3;
};

LinAlgVector operator+(const LinAlgVector &lhs, const LinAlgVector &rhs)
{
	LinAlgVector ret;
	ret.x1 = lhs.x1 + rhs.x1;
	ret.x2 = lhs.x2 + rhs.x2; 
	ret.x3 = lhs.x3 + rhs.x3;
	return ret;
}

int main()
{
	LinAlgVector a, b;
	a.x1 = 1; a.x2 = 2; a.x3 = 3;
	b.x1 = 1; b.x2 = 1; b.x3 = 1;
	
	LinAlgVector c = a + b; 
}
	\end{lstlisting}
	A \texttt{main} függvényben lévő értékadás ezzel ekvivalens: \texttt{c = operator+(a, b)}, így láthatjuk, hogy az operátorok túlterhelése gyakorlatilag a függvénytúlterhelés speciális esete. 
	\smallskip 
	
	Írjuk meg a \texttt{print} függvényt 3D vektorokra! A gyakran kiíratáshoz használt jobb shift operátor (\textit{right shift operator}), a \texttt{\<} is túlterhelhető. Mivel mi az \texttt{std::cout} változóval szeretnénk majd kiíratni, melynek típusa \texttt{std::ostream}, így a függvényünk első paramétere egy ilyen típus lesz, a második meg egy \texttt{LinAlgVector} típus.
	
	\begin{lstlisting}
/* ... */

std::ostream& operator<<(std::ostream& os, const LinAlgVector &l)
{
	os << l.x1 << ' ' << l.x2 << ' ' << l.x3;
	return os;
}

int main()
{
	/* ... */
	std::cout << c << std::endl; // 2 3 4
}
	\end{lstlisting}
	
	Feltűnhet, hogy a stream objektumra mutató referenciát a függvény végén vissza is adjuk, hogy tudjuk a kiíratást láncolni.
\end{document}
