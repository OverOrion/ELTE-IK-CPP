\documentclass[a4paper,11.5pt,table]{article}
\usepackage[textwidth=170mm, textheight=230mm, inner=20mm, top=20mm, bottom=30mm]{geometry}
\usepackage[normalem]{ulem}
\usepackage[utf8]{inputenc}
\usepackage[T1]{fontenc}
\PassOptionsToPackage{defaults=hu-min}{magyar.ldf}
\usepackage[magyar]{babel}
\usepackage{amsmath, amsthm,amssymb, paralist, tikz, multirow}
\usetikzlibrary{arrows, positioning}

\usepackage{listings}
\lstdefinestyle{customc}{
	belowcaptionskip=1\baselineskip,
	breaklines=true,
	frame=L,
	language=C++,
	showstringspaces=false,
	basicstyle=\ttfamily,
	%identifierstyle=\color{blue},
	keywordstyle=\bfseries\color{green!40!black},
	stringstyle=\color{orange},
	emph = {std, A, B, BDIT, Base, BaseOne, BaseTwo, C, CharT, Circle, Compr, ConstIterator, Cont, D, Deltoid, Derived, Derived2, DerivedLast, DerivedOne, DerivedTwo, Fact, Fiu, Foo, Greater, Hallgato, IT, InputIt, Iterator, Kor, Lany, Less, LessByX, LinAlgVector, List, Matrix, Negyszog, Negyzet, RAIT, Rombusz, S, Sielo, Sikidom, StringLength, T, T1, T2, Templ, U, Val, X, Y, Z, bad_cast, bidirectional_iterator_tag, const_iterator, deque, forward_iterator_tag, input_iterator_tag, iostream, istream, iterator, iterator_category, line_editor, list, map, multimap, multiset, ostream, pair, random_access_iterator_tag, set, string, strlen, strlenWrong, value_type, vector}, 
	emphstyle = \color{blue},
	tabsize = 4
}

\lstdefinestyle{customasm}{
	belowcaptionskip=1\baselineskip,
	frame=L,
	language=[x86masm]Assembler,
	basicstyle=\ttfamily,
	commentstyle=\itshape\color{purple!40!black},
}

\lstset{
	escapechar=@,
	style=customc,
	literate =  {á}{{\'a}}1 {é}{{\'e}}1 {í}{{\'i}}1 {ó}{{\'o}}1 {ú}{{\'u}}1
	{Á}{{\'A}}1 {É}{{\'E}}1 {Í}{{\'I}}1 {Ó}{{\'O}}1 {Ú}{{\'U}}1
	{ö}{{\"o}}1 {ü}{{\"u}}1 {Ö}{{\"O}}1 {Ü}{{\"U}}1
	{ű}{{\H{u}}}1 {Ű}{{\H{U}}}1 {ő}{{\H{o}}}1 {Ő}{{\H{O}}}1
	{€}{{\euro}}1 {£}{{\pounds}}1	
}

\lstset{escapechar=@,style=customc}

\usepackage{hyperref}

\begin{document}
	%%%%%%%%%%%RÖVIDÍTÉSEK%%%%%%%%%%
	\setlength\parindent{0pt}
	\def\<{<\hspace{0mm}<}
	
	\theoremstyle{definition}
	\newtheorem{note}{Megjegyzés}[subsection]
	%%%%%%%%%%%%%%%%%%%%%%%%%%%%%%%%%%%%%%%%%%%%%%%%%%%%%%%%%%%%%%%%%%%%%
	
	\begin{center}
		{\LARGE\textbf{C++}}
		
		{\Large Gyakorlat jegyzet}
		
		7. óra.
	\end{center}
	A jegyzetet \textsc{Umann} Kristóf készítette \textsc{Horváth} Gábor gyakorlatán. (\today)
	
	\section{Template}
	\subsection{Függvény template-ek}
	Térjünk vissza a régebben megírt swap függvényünkhöz.
	\begin{lstlisting}
void swap(int &a, int &b)
{
	int tmp = a;
	a = b;
	b = tmp;
}
	\end{lstlisting}
	Ahogy azt láttuk, túl tudjuk terhelni ezt a függvényt, hogy más típusú objektumokat is meg tudjunk cserélni.
	Azonban gyorsan megállapítható, hogy állandóan egy újabb overloadot létrehozni nem épp ideális megoldás. Ez a kisebb gond, a nagyobb az, hogy a kódismétlés áldozatai leszünk: ha bármi miatt megváltozna a \texttt{swap} belső implementációja (pl. találunk hatékonyabb megoldást), az összes létező swap függvényben meg kéne ejteni a változtatást. E probléma elkerülésére egy megoldás lehet, ha létrehozunk egy sablont, melynek mintájára a fordító maga tud generálni egy megfelelő függvényt.
	\begin{lstlisting}
template <typename T>
void swap(T &a, T &b)
{
	T tmp = a;
	a = b;
	b = tmp;
}
	\end{lstlisting}
	Az így implementált \texttt{swap} függvény egy \textit{template}, és a template paramétere (\texttt{T}) egy típus. Ez alapján a fordító már létre tud hozni hozni megfelelő függvényeket:
	\begin{lstlisting}
int main()
{
	int a = 2, b = 3;
	swap<int>(a, b);
	
	double c = 1.3, d = 7.8;
	swap<double>(c, d);
}
	\end{lstlisting}
	A fordítónak csak annyi dolga van, hogy minden \texttt{T}-t lecseréljen \texttt{int}-re, és már kész is a függvény. A fenti példában mi explicit megmondtuk a fordítónak, hogy \texttt{swap}-ot milyen template paraméterrel {példányosítsa} (\textit{instantiate}), azonban függvényeknél erre nem feltétlenül van szükség: a fordító tudja \texttt{a} és \texttt{b} típusát, így ki tudja találni hogy mit kell behelyettesítenie.
	\begin{lstlisting}
int main()
{
	int a = 2, b = 3;
	swap(a, b);
	
	double c = 1.3, d = 7.8;
	swap(c, d);
}
	\end{lstlisting}
	Ezt a folyamatot (amikor a fordító kitalálja a tempalte paramétert) \textbf{template paraméter dedukciónak} (\textit{template parameter deduction}) hívjuk.
	
	\medskip
	Nem csak típus lehet template paraméter -- bármi ami \textbf{nem} karakterlánc literál vagy lebegőpontos szám.
	\begin{lstlisting}
template <typename T, int ArraySize>
int arraySize(const T (&array)[ArraySize])
{
	return ArraySize;
}

int main()
{
	int i[10];
	std::cout << arraySize(i) << std::endl; //10
}
	\end{lstlisting}
	A fenti kód a 3. gyakorlat végén tett megjegyzésből lehet ismerős. Jól demonstrálja a template paraméter dedukciót.
	\subsection{Osztály template-ek}
	Nem csak függvények, osztályok is lehentek template-ek melyen nagyon hasonlóan működnek.
	\begin{lstlisting}
#include <iostream>

template <typename T>
struct X
{
	void f()
	{
		T t;
		t.foo();
	}
};
struct Y
{
	void bar() {}
};
int main() {}
	\end{lstlisting}
	Ez a kód úgy tűnhet, hogy nem fog lefordulni, lévén mi soha semmilyen \texttt{foo} tagfüggvényt nem írtunk, azonban mégis le fog. Ez azért van, mert a template osztályok (és függvények) csak sablonok, amiből aminek alapján a fordító generálhat egy konkrét osztályt (vagy függvényt), és mivel sose példányosítottuk, fordítás után az \texttt{X} template osztály nem fog szerepelni a kódban. Szintaktikus ellenőrzést végez a fordító, (pl. zárójelek be vannak-e zárva, pontosvessző nem hiányzik-e stb.), de azt, hogy van-e olyan \texttt{T} típus, ami rendelkezik \texttt{foo()} függvénnyel, már nem.
	
	\smallskip
	Példányosítsuk az \texttt{X} osztályt \texttt{Y}-nal!
\begin{lstlisting}
int main()
{
	X<Y> x;
}
\end{lstlisting}
	Ekkor már azt várnánk hogy fordítási hibát dobjon a fordító, hisz \texttt{Y}-nak nincs \texttt{foo()} metódusa, azonban mégis gond nélkül lefordul, mivel az \texttt{f()} tagfüggvényt nem hívtuk meg, így nem is példányosult az osztályon belül.
	\begin{lstlisting}
int main()
{
	X<Y> x;
	x.f();
}
	\end{lstlisting}
	Itt már végre kapunk fordítási hibát, mert példányosul \texttt{f()}. Ez jól mutatja, hogy a template-ek lusták, és csak akkor példányosulnak, ha ,,nagyon muszáj''.
	\medskip
	
	A template-eknek adhatunk meg alapértelmezett értéket.

\begin{lstlisting}
template <typename T = void> //alapértelmezett paraméter
struct X {/* ... */};

struct Y {/* ... */};

int main()
{
	X<Y> x;
	X<> x2;
}
\end{lstlisting}
	Ilyenkor nem szükséges megadni template paramétert (mely esetben értelemszerűen \texttt{X<> == X<void>}). 
	
	\smallskip
	Ahogyan az említve volt korábban, szinte bármi lehet template paraméter, akár egy másik template is.

\begin{lstlisting}
template <typename T>
struct X {/* ... */};

struct Y {/* ... */};

template <template <typename> class Templ>
struct Z
{
	Templ<int> t;
};

int main()
{
	Z<X> z;
}
\end{lstlisting}
	Fent \texttt{Templ} egy olyan template, aminek a template paramétere egy típus. Így \texttt{Z}-nek a template paramétere egy olyan template, aminek a template paramétere egy típus. Mivel \texttt{X} egy template (és template paramétere egy típus), így megadható \texttt{Z}-nek template paraméterként.
	\begin{note}
		Fent a template paraméter listában \texttt{typename} helyett \texttt{class} szerepel. Ezek gyakorlatilag ekvivalensek, mind a kettő azt jelenti, hogy az adott paraméter típus (bár a \texttt{typename} beszédesebb).
	\end{note}
	
	A fenti példákban mindig egy default konstruktort használunk, amikor objektumokat hoztunk létre. Helyes lenne-e az, ha explicit módon kíírnánk a zárójeleket (hangsúlyozva a default konstruktor hívást)?
\begin{lstlisting}
int main()
{
	X<Y> x();
	X<> y2();
	Z<X> z();
}
\end{lstlisting}
	A kód helyesen lefordul, de nem ugyanaz, mintha nem lenne ott a zárójel. Mivel a c++ nyelvtana nem egyértelmű, más kontextusban ugyanaz a kódrészlet mást jelenthet (egyik legegyszerűbb példa a \texttt{static} kulcsszó), így meg kellett alkotni egy olyan szabályt, miszerint amit deklarációként lehet értelmezni, azt deklarációként \textbf{kell} értelmezni. Igy ezek függvénydeklarációk lesznek: Az első esetben például egy olyan függvényt deklarálunk, melynek neve \texttt{x}, \texttt{X<Y>}-al tér vissza és nem vár paramétert. 
	
	Így ha default konstruktort szeretnék meghívni, semmilyen zárójelt nem szabad használni.
	\begin{note}
		C++11ben lehet \texttt{()} helyett \{\} zárójelet alkalmazni konstruktorhívásnál, így ez a probléma nem fordulhat elő. pl: \texttt{X<Y> x\{\};}
	\end{note}
	
	A template-ek paramérének ismertnek kell lennie fordítási időben.
	\begin{lstlisting}
template <int N>
void f() {}

int main()
{
	int n;
	std::cin >> n;
	f<n>(); //hiba, n nem ismert fordítási időben
}
	\end{lstlisting}
	Ez nyilvánvaló, hisz a template-eknek az a funkciója, hogy a fordító generáljon belőlük példányokat, és a fordítási idő végeztével erre nincs lehetőség.
	\begin{note}
		Fontos még, hogy a template-ek nagyon megnövelik a fordítási időt, így nem mindig éri meg egy olyan függvényt is template-ként megírni, melyet nem feltétlenül muszáj.
	\end{note}
	\subsection{Template specializáció}
	Néha szeretnénk, hogy bizonyos speciális behelyettesítéseknél más legyen az implementáció mint az alap sablonban. Ilyenkor szokás \textbf{specializációkat} (\textit{template specialization}) létrehozni:
	
	\begin{lstlisting}
template <class T>
struct A
{
	A()	{ std::cout << "general A" << std::endl; }
};

template <> //template specializáció
struct A<int>
{
	A() { std::cout << "special A" << std::endl; }
};

template <class T>
void f() { std::cout << "general f" << std::endl; }

template<> //template specializáció
void f<int>() { std::cout << "special f" << std::endl; }

int main() 
{
	A<std::string> a1; //general A
	f<std::string>(); //general f
	A<int> a2; //special A
	f<int>(); //special f
}
	\end{lstlisting}
	Mind \texttt{A} osztályhoz, mind \texttt{f} függvényhez létrehoztunk egy specializációt arra az esetre, ha a template paraméterként \texttt{int}-et kapnak. Számos okunk lehet arra hogy ezt tegyük: a standard könyvtár megfényesebb példája az \texttt{std::vector} osztály, mely egy template, és van template specializációja \texttt{bool} esetre.
	\begin{note}
		Az \texttt{std::vector<bool>} számos optimalizációkat tartalmazhat (persze nem feltétlenül, hisz ez implementáció függő): általában nem \texttt{bool}-okban tárolja az adatokat, hanem bitekben. Sajnos azonban ez hátrányokkal is jár, például hogy a \texttt{[]} operátor érték és nem referencia szerint ad vissza -- bár valóban hatékonyabb, sok szempontból fejfájást okozhat a használata, így c++17-ben ez a specializáció nem fogja a szabvány részét képezni.
	\end{note}
	Írjunk faktoriális számoló algoritmus template-ek segítségével!
	\begin{lstlisting}
template<int N>                           
struct Fact 
{                             
	static const int val = N*Fact<N-1>::val;
};

template<>                                
struct Fact<0>
{                          
	static const int val = 1;               
};               

int main() 
{                                          
	std::cout << Fact<5>::val << std::endl; //120
}
	\end{lstlisting}
	\texttt{Fact} 4szer példányosul: \texttt{Fact<5>, \ldots, Fact<1>,} majd a legvégén az általunk specializált \texttt{Fact<0>}-t hívja meg.
	\begin{note}
		Ahogyan ezt korábban megállapítottunk, egy \textbf{konstans} osztályszintű változót függvénytörzsön belül is inicializálhatunk.
	\end{note}
	Ez fel is hívja a figyelmet a template-ek veszélyeit statikus változók használatakor.
	
	\begin{lstlisting}
template <class T>
class A
{
	static int count;
public:
	A()
	{
		std::cout << ++count << ' ';
	}
};

template <class T>
int A<T>::count = 0;

int main() 
{
	for(int i = 0; i<5; i++)
	{
		A<int> a;
		A<double> b;
	}
}
	\end{lstlisting}
	Kimenet: \texttt{1 1 2 2 3 3 4 4 5 5}
	
	Bár arra számítanánk, hogy 1-től 10ig lesznek a számok kiírva, ne felejtsük, hogy itt két teljesen különböző osztály fog létrejönni: \texttt{A<int>} és \texttt{A<double>}, így a \texttt{count} adattag hiába osztályszintű, 2 teljesen különböző példánya lesz ennek is: \texttt{A<int>::count} és \texttt{A<double>::count}.
	
	\subsection{Dependent scope}
	
	Lehetőségünk van arra hogy osztályon belül deklaráljunk még egy osztályt. Bár erről bővebben a következő órai jegyzetben lesz szó, egy igen fontos problémát vet fel.
	\begin{lstlisting}
class A
{
public:
	class X {};
};

void f(A a)
{
	A::X x;
}

int main()
{
	A a;
	f(a);
}
	\end{lstlisting}
	Ezzel semmi probléma nincs. Legyen A egy template osztály!
	\begin{lstlisting}
template <class T>
class A
{
public:
	class X {};
};

template <class T>
void f(A<T> a)
{
	A<T>::X x;
}

int main()
{
	A<int> a;
	f(a);
}
	\end{lstlisting}
	Itt máris bajba jutottunk, a fordító azt a hibát fogja jelezni, hogy \texttt{X} egy un. \textbf{dependent scope}-ban van. Ez azt jelenti, hogy attól függően, milyen template paraméterrel példányosítjuk \texttt{A}-t, \texttt{X}-nek lehet más a jelentése. Az alábbi kód ezt jól demonstrálja:
	\begin{lstlisting}
template <typename T>
struct A
{
	class X{};
};

template <>
struct A <int>
{
	static int X;
};

int A<int>::X = 0;

template <typename T>
void f()
{
	A<T>::X;
}
	\end{lstlisting}
	Itt az \texttt{f} függvényben vajon mi lesz \texttt{A<T>::X}? A válasz az hogy nem tudni, hisz ha \texttt{int}-el példányosítunk akkor statikus adattag, ha bármi mással, akkor meg egy típus. Ezért kell a fordítónak biztosítani, hogy a template paramétertől függetlenül garantáltan típust fog oda kerülni. Ezt a \texttt{typename} kulcssszóval tehetjük meg.
	\begin{lstlisting}
template <typename T>
void f()
{
	typename A<T>::X;
}
	\end{lstlisting}
	A \texttt{typename} garantálja a fordítónak, hogy bármi is lesz \texttt{T}, \texttt{A<T>::X} mindenképpen típus lesz. Ha mégis olyan template paramétert adunk meg, aminél ez nem teljesülne (ez esetben \texttt{T = int}) akkor fordítási idejű hibát kapunk.
	\begin{note}
		A fordító általában szokott szólni, hogy a \texttt{typename} kulcsszó hiányzik.
	\end{note}
	\begin{note}
		A dependent scope problémája nem csupán az osztályon belüli osztályokra érvényes. Nemsokára meglátjuk, hogy a \texttt{typedef} kulcsszó is ide tud vezetni.
	\end{note}
	Így viszont felmerülhet a kérdés hogy van-e szükség \texttt{typename} kulcsszóra, ha egy \texttt{std::vector<int>::iterator} típusú objektumot akarunk létrehozni (\texttt{std::vector<int>::iterator vit;}). A válasz az hogy nem, hisz ha konkrétan megadjuk a típust, amellyel példányosítanánk, akkor a fordító arra a konkrét típusra vissza tudja keresni, hogy \texttt{std::vector<int>::iterator} típus-e, vagy sem.
	
\end{document}
